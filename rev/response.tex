\documentclass[12pt]{article}
\usepackage[utf8]{inputenc}

\usepackage{color}

\usepackage{lmodern}
\usepackage{amssymb,amsmath}

\newcommand{\rR}{\mbox{$r$--$\cal R$}}
\newcommand{\RR}{\ensuremath{{\cal R}}}
\newcommand{\RRhat}{\ensuremath{{\hat \cal R}}}
\newcommand{\Rx}[1]{\ensuremath{{\cal R}_{#1}}} 
\newcommand{\Ro}{\Rx{0}}
\newcommand{\Reff}{\Rx{\mathit{eff}}}
\newcommand{\Tc}{\ensuremath{C}}

\newcommand{\rev}{\subsection*}
\newcommand{\revtext}{\textsf}
\setlength{\parskip}{\baselineskip}
\setlength{\parindent}{0em}

\newcommand{\comment}[3]{\textcolor{#1}{\textbf{[#2: }\textsl{#3}\textbf{]}}}
\newcommand{\jd}[1]{\comment{cyan}{JD}{#1}}
\newcommand{\swp}[1]{\comment{magenta}{SWP}{#1}}
\newcommand{\dc}[1]{\comment{blue}{DC}{#1}}
\newcommand{\jsw}[1]{\comment{green}{JSW}{#1}}
\newcommand{\hotcomment}[1]{\comment{red}{HOT}{#1}}

\begin{document}

\noindent Dear Editor:

Thank you for the chance to revise and resubmit our manuscript. 
Below please find our responses to reviewers.

\rev{Reviewer \#1}

\revtext{Summary: This article describes the implications of asymptomatic transmission of COVID-19 for estimating the overall growth of the outbreak during the exponential phase and frames these results in terms of the generation time of asymptomatic cases relative to symptomatic cases.  The central result is intuitive and, as a modeler, I have frequently found myself trying to explain this tendency to colleagues over the past few weeks.  It is useful to see a rigorously developed example here for helping to frame the conversation and guide adjustments to ongoing modeling work.  However, some of the methods require additional clarification and additional discussion is needed to help increase the accessibility of the findings to a broader audience. Specific comments follow.}

Thank you very much!

\revtext{It might be useful to clarify explicitly in the main text that differences in generation time can occur irrespective of differences in transmissibility (as is shown in equations S17 and S18).  For example, asymptomatic cases may be equally infectious to symptomatic cases in the early stages of their infection, but simply recover more quickly, which would lower the generation time for symptomatic vs. asymptomatic cases.}  

We have clarified this point in the following sentence:

``The differences in the generation-interval distributions between asymptomatic and symptomatic cases can be caused by the differences in the natural history of infection irrespective of their transmissibility:
Individuals with asymptomatic infections may recover faster and have short generation intervals, or have persistent infection and long generation intervals.''

\revtext{It might be useful to also note in the discussion how differences in infectiousness might layer on top of differences in duration to change estimates of R0.} 

Differences in infectiousness is captured by the differences in their intrinsic reproduction numbers $\mathcal R_a$ and $\mathcal R_s$, and therefore their intrinsic proportion of transmission $z$ and $1-z$; this is already explained in the paper.

\revtext{At present, I think your method implicitly assumes that transmissibility is the same for symptomatic and asymptomatic cases as their r value is the same.}

We do not assume that their transmissibility is the same; the differences in their transmissibility is modeled by the differences in their intrinsic reproduction numbers $\mathcal R_a$ and $\mathcal R_s$ as well as their generation-interval distributions $g_a(\tau)$ and $g_s(\tau)$. The exponential growth rate $r$ is the resulting growth rate due to combination of asymptomatic and symptomatic transmission. This growth rate is different from what we would get if we only have asymptomatic or symptomatic individuals. We added the following sentence to clarify this point:

``During the period of exponential growth, we assume $S$ remains nearly constant, and $i(t)$ is proportional to $\exp(r t)$;
here, the observed exponential growth rate $r$ is an average of the exponential growth rates we would observe if there were only asymptomatic ($p=1$) or symptomatic ($p=0$) cases.''

\revtext{Please be explicit in the methods section about how the naive estimation of R0 is done (presumably using equation 10) and also provide an expanded equation for the 'corrected' estimation using the stratified generation times. Right now, the clearest description is in the caption of figure 1.  I think you mean to say that the naive estimate of R0 is based on the generation time of symptomatic cases using the Lotka-Voltera estimation method (equation 10) and assumes that this interval is similar for symptomatic and asymptomatic cases.}

We have added the following sentences:

``We infer values of $q$ using Eq.~(8) and ${\cal{R}}_0$ using the Euler-Lotka equation:
\begin{equation}
\frac{1}{\mathcal R_0} = \int \exp(-r \tau) \left(z g_a(\tau) + (1-z) g_s(\tau)\right) \mathrm{d} \tau.
\end{equation}
We compare this with the naive estimate of the basic reproduction number that assumes that the generation-interval distributions of the asymptomatic and symptomatic cases are identical:
\begin{equation}
\frac{1}{\mathcal R_{\tiny \textrm{naive}}} = \int \exp(-r \tau) g_s(\tau) \mathrm{d} \tau.
\end{equation}''

\revtext{Top of page 5: provide an example about how we might expect the distribution of generation-intervals to vary between symptomatic vs. asymptomatic infections}

We explain the potential differences earlier as suggested:

``The differences in the generation-interval distributions between asymptomatic and symptomatic cases can be caused by the differences in the natural history of infection irrespective of their transmissibility:
Individuals with asymptomatic infections may recover faster and have short generation intervals, or have persistent infection and long generation intervals.''

\revtext{Figure 1:
-In the caption, please state 'For both panels, the triangle…'
-Panel B) Please rename the dashed line something like 'naive R0' so that it is clear that the contours reflect the corrected reproduction number and we understand why the regions on the left side of the dashed line says underestimated and the right side says overestimated.  It took me a while to realize that the dashed line was the naive estimate so I thought that the colors were reversed.}

Done. We also added the following sentence in the figure caption:

``Dashed line represents the naive estimate that assumes $\bar G_a = \bar G_s$. ''

\revtext{In the discussion, the authors could note that inaccurate estimation of R0 could impact estimates of control efforts needed to interrupt (or appreciably reduce) transmission.}

We have added the following sentence:

``The biases in the estimates of ${\cal{R}}_0$ will necessarily bias the estimates of the amount of intervention required to control the epidemic.''

\revtext{Although not strictly necessary, it might be useful for the authors to provide a table defining the parameters used in the equations somewhere, as at times I had to go back and forth through the article to remind myself of how things were defined.  This could be a supplementary table.}

Done. We introduce the table when we introduce the model:

``Neglecting births and loss of immunity on the time scale of the outbreak, the dynamics of susceptibles and incidence are (see Table~S1 for parameter definitions)''

\rev{Reviewer \#2} 

\revtext{This is a clearly-written, insightful manuscript that shows mathematically and through simulation how R0 estimates for SARS-CoV-2 can be biased if asymptomatic individuals are not taken into consideration. Specifically, if asymptomatic individuals differ from symptomatic individuals in their generation interval distribution, then R0 could be considerably over- or under-estimated. This is fundamentally because estimation of R0 values from observed intrinsic growth rates of an epidemic relies on specification of the generation interval (the time between infection and onward transmission). If these differ between symptomatic and asymptomatically infected individuals (which they very much may), then R0 estimates that assume the generation interval is that of symptomatic individuals could be severely biased.
I have only a couple of very minor suggestions. This manuscript, overall, is excellent and very clearly presented.}

Thank you very much!

\revtext{1. This may be overkill, but near equation (1), explicitly state that $i(t) = i_a(t) + i_s(t)$?}

Done.

``Neglecting births and loss of immunity on the time scale of the outbreak, the dynamics of susceptibles and incidence are:
\begin{eqnarray}
\dot{S}&=&-i(t), \\
i(t)&=&\mathcal R_a S(t) \int_0^\infty i_a(t-\tau) g_a(\tau) \mathrm{d}\tau + \mathcal R_s S(t) \int_0^\infty i_s(t-\tau) g_s(\tau) \mathrm{d}\tau,
\end{eqnarray}
where $i(t) = i_a(t) + i_s(t)$.''

\revtext{2. Indicate below equation 10 that the ode model in the supplement is an SEIR model}

Done.

``In Supplementary Materials, we also use an ordinary differential equation model (SEIR model) including both asymptomatic and symptomatic cases to give a concrete example of how differences in generation intervals affect both $q$ and estimates of $\mathcal R_0$.''

\revtext{3. Figure 1A: it may be more helpful to see the difference between realized proportion and intrinsic proportion as the heatmap color, rather than the realized proportion}

We have added the figure to the supplementary file with the following sentence in the main text:

``In Figure~S1, we present the same figure but showing differences between the realized and the intrinsic proportion of asymptomatic transmission, $q-z$.''

\end{document}
