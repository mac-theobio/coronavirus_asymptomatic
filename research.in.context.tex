\section*{Research in context}

\subsection*{Evidence before this study}

We searched PubMed for articles published up to March 15, 2020 using terms ``2019-nCoV'', ``novel coronavirus'', ``COVID-19'', ``SARS-CoV-2'' AND ``asymptomatic''. We found several studies describing asymptomatic infection and one study suggesting transmission by an asymptomatic carrier. 
We found one study, published on March 16, 2020, estimating the transmission rate of asymptomatic carriers of COVID-19, their prevalence, and the proportion of secondary cases caused by them.
However, we found no studies describing the effect of asymptomatic transmission in estimating the basic reproduction number of COVID-19.
To our knowledge, this is the first study to describe the effects of differences in time scales of asymptomatic and symptomatic transmission on disease dynamics.

\subsection*{Added value of this study}

We use a mathematical model to assess how the time scale of asymptomatic transmission affects the epidemic potential for COVID-like pathogens.
We compare the basic reproduction number and the proportion of secondary cases caused by asymptomatic individuals during the initial exponential growth period across a wide range of assumptions about asymptomatic transmission.
These assumptions broadly fall under two opposite scenarios: asymptomatic individuals can have fast clearance (and therefore, faster transmission) or slow viral replication (and therefore, persistent infection and slower transmission).
If asymptomatic transmission is slower than symptomatic transmission, the basic reproduction number can be underestimated if asymptomatic transmission is not explicitly taken into account.
If asymptomatic transmission is faster than symptomatic transmission, asymptomatic transmission may still account for a large fraction of secondary infection.

\subsection*{Implications of all the available evidence}

The prevalence of asymptomatic carriers of COVID-19 and their transmission potential remain unclear. 
Future research should prioritize characterizing the course of infection for asymptomatic cases and estimating their prevalence.
%% Current estimates that do not account for asymptomatic transmission may be systematically biased and should be refined accordingly.
