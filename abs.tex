{\footnotesize We assess the impact of asymptomatic transmission on epidemic potential of novel respiratory pathogens (like COVID-19) --
as measured both by the basic reproduction number (i.e., the expected number of secondary cases generated by an average primary case in a fully susceptible population) and the fraction of new secondary cases attributable to asymptomatic individuals. 
We show that the impact of asymptomatic transmission depends on generation intervals (i.e., time between when an individual is infected and when that individual infects another person).
If the generation intervals of asymptomatic cases differs from that of symptomatic cases, then estimates of the basic reproduction number which do not explicitly account for asymptomatic cases may be systematically biased. 
Specifically, if asymptomatic cases have a shorter generation interval than symptomatic cases, ${\cal{R}}_0$ will be over-estimated, and if they have a longer generation interval, ${\cal{R}}_0$ will be under-estimated.
The analysis provides a rationale for assessing the duration of asymptomatic cases of COVID-19 in addition to their prevalence in the population.
}
