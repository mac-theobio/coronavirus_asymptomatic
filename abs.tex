{\footnotesize We assess the impact of asymptomatic cases on epidemic potential of COVID-19 (and COVID-19 like pathogens) -- as measured both by the basic reproduction 
number (i.e., the expected number of secondary cases generated by an average primary case in a fully susceptible population) and the fraction of new secondary cases attributable to asymptomatic
individuals. As we show, the impact of asymptomatic
cases depends on their generation intervals (i.e., time between when an individual is infected and when that individual infects another person).
If the generation intervals of asymptomatic cases differs from that 
of symptomatic cases, then estimates of the basic reproduction number which do not explicitly account for asymptomatic cases may be systematically biased. Specifically, if asymptomatic cases have a shorter generation interval, 
${\cal{R}}_0$ will be over-estimated, and if they have a longer generation interval, ${\cal{R}}_0$ will be under-estimated.
%If infectiousness of asymptomatic cases is relatively short/long in duration compared
%to symptomatic cases, then
%estimates of the basic reproduction number may be over/under-estimates
%compared to estimates only attributing secondary cases to symptomatic cases.
The analysis provides a rationale for assessing the duration of asymptomatic cases 
of COVID-19 in addition to their prevalence in the population.
}
