{\footnotesize We analyze a mechanistic model of a coronavirus-like pathogen
to assess the impact of asymptomatic cases on epidemic potential
-- as measured both by the basic reproduction
number and the fraction of new secondary cases attributable to asymptomatic
individuals. As we show, the impact of asymptomatic
cases depends on the generation intervals of asymptomatic cases.  
If the generation intervals of asymptomatic cases differs than that 
of symptomatic cases, then estimates of the basic reproductive ratio which do not explicitly account for asymptomatic cases may be systematically biased. Specifically, if asymptomatic cases have a shorter generation interval, 
${\cal{R}}_0$ will be over-estimated, and if they have a longer generation interval, ${\cal{R}}_0$ will be under-estimated.
%If infectiousness of asymptomatic cases is relatively short/long in duration compared
%to symptomatic cases, then
%estimates of the basic reproduction number may be over/under-estimates
%compared to estimates only attributing secondary cases to symptomatic cases.
The analysis provides a rationale for assessing the duration of asymptomatic cases 
of COVID-19 in addition to their prevalence in the population.
}
