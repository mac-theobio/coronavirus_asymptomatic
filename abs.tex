{\footnotesize 

The role of asymptomatic carriers to transmit virus poses challenges for control of the COVID-19 pandemic. 
Study of asymptomatic transmission and implications for surveillance and disease burden is ongoing. 
There has been little study of the implications of asymptomatic transmission on dynamics of disease.
We use a mathematical framework to evaluate expected effects of asymptomatic transmission on the basic reproduction number ${\cal{R}}_0$ (i.e., the expected number of secondary cases generated by an average primary case in a fully susceptible population) and the fraction of new secondary cases attributable to asymptomatic individuals.
If the generation-interval distribution of asymptomatic transmission differs from that of symptomatic transmission, then estimates of the basic reproduction number which do not explicitly account for asymptomatic cases may be systematically biased. 
Specifically, if asymptomatic cases have a shorter generation interval than symptomatic cases, ${\cal{R}}_0$ will be over-estimated, and if they have a longer generation interval, ${\cal{R}}_0$ will be under-estimated.
Estimates of the realized proportion of asymptomatic transmission during the exponential phase also depend on asymptomatic generation intervals.
Our analysis shows that understanding the temporal course of asymptomatic transmission can be important for assessing the importance of this route of transmission, and for disease dynamics. This provides an additional motivation for investigating both the prevalence and relative duration of asymptomatic transmission. 

\subsection*{Keywords}

SARS-CoV-2, COVID-19, coronavirus disease, asymptomatic transmission, basic reproduction number

\subsection*{Funding}

This work was supported, in part, by support from the Army Research Office to JSW (W911NF1910384), and support from the Canadian Institutes of Health Research to JD.}
