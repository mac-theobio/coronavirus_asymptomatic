{\footnotesize 
\section*{Background}

The potential for asymptomatic carriers to transmit the virus poses a challenge for controlling the coronavirus disease 2019 (COVID-19) pandemic but their roles remain unknown.
We assess the impact of asymptomatic transmission on epidemic potential of novel respiratory pathogens (like COVID-19).

\section*{Methods}

We develope a mathematical model for comparing asymptomatic and symptomatic transmission based on their generation intervals (i.e., time between when an individual is infected and when that individual infects another person).
We measure the effect of asymptomatic transmission on the basic reproduction number (i.e., the expected number of secondary cases generated by an average primary case in a fully susceptible population) and the fraction of new secondary cases attributable to asymptomatic individuals, across a wide range of assumptions about asymptomatic transmission.

\section*{Findings}

If the generation-interval distribution of asymptomatic transmission differs from that of symptomatic transmission, then estimates of the basic reproduction number which do not explicitly account for asymptomatic cases may be systematically biased. 
Specifically, if asymptomatic cases have a shorter generation interval than symptomatic cases, ${\cal{R}}_0$ will be over-estimated, and if they have a longer generation interval, ${\cal{R}}_0$ will be under-estimated.
We also show that as the length of asymptomatic generation intervals increase, estimates of the realized proportion of asymptomatic transmission during the exponential phase of the epidemic decrease.

\section*{Interpretation}

Our analysis provides a rationale for assessing the duration of asymptomatic cases of COVID-19 in addition to their prevalence in the population.

\section*{Funding}

This work was supported, in part, by support from the Army Research Office to JSW (W911NF1910384).
}
